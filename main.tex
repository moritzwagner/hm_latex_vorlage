% Vorgabe der Text- und Papeiergröße und den Titel auf eigene Seite
\documentclass[a4paper,12pt,twoside,notitlepage, letterpaper]{article}
% Bilderimport
\usepackage{graphicx}
\usepackage{float}
% Deutsche Localization
\usepackage[ngerman, english]{babel}
% Erweiterte Refernezierungsmöglichkeiten
% hidelinks = Keine rote Umrandung
\usepackage[hidelinks]{hyperref}
% Times New Roman Schriftart
\usepackage{times}
% Fix für deutsche Schriftzeichen und andere Non-ASCII-Zeichen
\usepackage[T1]{fontenc}
% Zitatzeichensetzung
\usepackage[%
left = \glqq,%
right = \grqq,%
leftsub = \glq,%
rightsub = \grq%
]{dirtytalk}
% Highlight
\usepackage{soul}
\DeclareRobustCommand{\hlyellow}[1]{\sethlcolor{yellow}\hl{#1}}
\usepackage{color}
% Tabellenzelle über mehrere Zeilen
\usepackage{multirow}
% Mehrere Spalten
\usepackage{multicol}
% Zitierung
\usepackage[backend=bibtex,style=authoryear-icomp, maxcitenames=2]{biblatex}
% Letze Seite
\usepackage{lastpage}
% Titel Anpassungen
\usepackage{titling}

% Laden der Literaturquellen
\addbibresource{sources.bib}

% Verbesserte Tabellen
\usepackage{tabularx}

% Um Tabellenbreite zu bestimme
\usepackage{array}

% Abkürzungsverzeichnis und Glossar hinzufügen
% Abkürzungsverzeichnis

\usepackage{acro}

\acsetup{make-links}

\DeclareAcronym{XAI}{
    short=XAI,
    long=eXplainable Artificial Intelligence,
    foreign=Erklärbare Künstliche Intelligenz
}
% Glossar

\usepackage[style=indexgroup, numberedsection, automake]{glossaries-extra}

\makeglossaries

\newglossaryentry{ki}{
    name=Künstliche Intelligenz,
    description={Computersystem, welches intelligentes, menschliches Denken emulieren kann und selbständig durch Analyse seiner Umgebung Maßnahmen identifizieren bzw. Entscheidungen treffen kann.}
}


% Styling-Anpassungen importieren
% Seitenlayout

% Zeilenabstand
\usepackage{setspace}
\setstretch{1.05}

% Keinen Einzug bei neuem Absatz
\setlength{\parindent}{0pt}
\setlength{\parskip}{6pt}

% Seitenränder
\usepackage[left=1.65cm, right=1.65cm, top=2.0cm, bottom=1.78cm]{geometry}
% Kopf- und Fußzeile

\usepackage{fancyhdr}

\pagestyle{fancy}

% Kopfzeile
\fancyhead{} % clear all header fields
% Kopfzeile links gerade Seite
\fancyhead[LE]{\leftmark}
% Kopfzeile rechts ungerade Seite
\fancyhead[RO]{Header bearbeiten in style/header-footer.tex}
% Kopfzeile links ungerade Seite (Seitenzahl [Seite von Seitenanzahl])
\fancyhead[LO]{Seite \thepage{} von \pageref{LastPage}}
% Kopfzeile rechts gerade Seite (Seitenzahl [Seite von Seitenanzahl])
\fancyhead[RE]{Seite \thepage{} von \pageref{LastPage}}

% Kopfzeile erste Seite
\fancypagestyle{plain}{
    \fancyhf{}
    \fancyhead[L]{Seite \thepage{} von \pageref{LastPage}}
    \fancyhead[C]{Modulname}
    \fancyhead[R]{Organisation}
    \renewcommand{\headrulewidth}{0.4pt} 
}

% Seitenlayout

% Zeilenabstand
\usepackage{setspace}
\setstretch{1.05}

% Keinen Einzug bei neuem Absatz
\setlength{\parindent}{0pt}
\setlength{\parskip}{6pt}

% Seitenränder
\usepackage[left=1.65cm, right=1.65cm, top=2.0cm, bottom=1.78cm]{geometry}
\makeatletter
\def\@maketitle{%
  \newpage
%  \null% DELETED
%  \vskip 2em% DELETED
  \begin{center}%
  \let \footnote \thanks
    {\LARGE \@title \par}%
    \vskip .5em%
    {\large
      \lineskip .1em%
      \begin{tabular}[t]{c}%
        \@author
      \end{tabular}\par}%
    \vskip 1em%
    {\small \@date}%
  \end{center}%
  \par
  \vskip 1.5em}
\makeatother
% Redefine \thanks{text} 
\makeatletter
\def\thanks#1{\protected@xdef\@thanks{\@thanks
        \protect\footnotetext{#1}}}
\makeatother
\DeclareCiteCommand{\cite}
  {\usebibmacro{prenote}}
  {\usebibmacro{citeindex}%
   \textsc{\printtext[brackets]{\usebibmacro{cite}}}}
  {\multicitedelim}
  {\usebibmacro{postnote}}
\usepackage[compact]{titlesec}
\usepackage{color}
\usepackage{xcolor}
\usepackage{titling}

\definecolor{customblue}{RGB}{0,0,153}
\makeatletter
\beforetitleunit=1ex\@plus.6ex\@minus.12ex
\makeatother

\titlespacing*{\section}{0pt}{*1}{*1}
\beforetitleunit= 0.1cm
\titleformat{\section}
{\color{customblue}\normalfont\Large\bfseries}
{\color{customblue}\thesection}{1em}{}

%Textfüller
\usepackage{lipsum}

\begin{document}
\thispagestyle{fancy}

% Titel
\title{Titel der Studienarbeit}
% Autoren
\author{
    Vorname Nachname
    \thanks{Resume: Vorname Nachname \href{mailto:mail@organisation.edu}{mail@organisation.edu} ist Student an der X Organisation.}%
    \thanks{Diese Studienarbeit wurde im Rahmen des Moduls \glqq Modulname\grqq\ erstellt.}% 
}
% Datum
\date{München, 01.01.2024}
\maketitle

% Abstract
\vspace{-1cm} % Abstand nach Author entfernen
% Abstract in der Datei content/0_Abstract.tex erstellen
\begin{abstract}
    Diese Studienarbeit beschäftigt sich mit dem Thema \glqq Titel der Studienarbeit\grqq. 
\end{abstract}

% Keywords

\providecommand{\keywords}[1]
{
  \small	
  \textbf{\textit{Schlüsselworte---}} #1
}
\keywords{
    Erstes Schlüsselwort, Zweites Schlüsselwort, Drittes Schlüsselwort, Viertes Schlüsselwort, Fünftes Schlüsselwort
}


\begin{multicols}{2}
    % Inhalt für die Arbeit hier definieren
% Reihenfolge gibt die Reihenfolge der Überschriften im Dokument vor

\section{Kapitel 1}

Dieser Artikel wurde referenziert \cite{bauer_explained_2023}. Um eine Abkürzung zu verwenden, kann \ac{XAI} verwendet werden. Das Glossar kann mit \gls{ki} aufgerufen werden.

Eine Grafik kann so eingebunden werden:
\begin{figure}[H]
    \centering
    \includegraphics[width=0.5\textwidth]{assets/placehoder.png}
    \caption{Beispielgrafik}
    \label{fig:example}
\end{figure}

\lipsum[1-2]
\section{Kapitel 2}
\label{sec:Kapitel2}
% Platzhaltertext
\lipsum[1-1]

\subsection{Unterkapitel 2.1}
\label{subsec:Unterkapitel2.1}
% Platzhaltertext
\lipsum[2-2]

\subsubsection{Unterunterkapitel 2.1.1}

\lipsum[3-3]
\section{Kapitel 3}
\label{sec:Kapitel3}
% Platzhaltertext
\lipsum[4-4]

    \appendix

    \printbibliography[title=Literaturverzeichnis]

    \printglossary

    \section{Abkürzungsverzeichnis}
    \printacronyms[heading=none]

    \section{Abbildungsverzeichnis}
    \listoffigures
\end{multicols}

\end{document}